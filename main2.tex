% ****** Start of file apssamp.tex ******
%
%   This file is part of the APS files in the REVTeX 4.1 distribution.
%   Version 4.1r of REVTeX, August 2010
%
%   Copyright (c) 2009, 2010 The American Physical Society.
%
%   See the REVTeX 4 README file for restrictions and more information.
%
% TeX'ing this file requires that you have AMS-LaTeX 2.0 installed
% as well as the rest of the prerequisites for REVTeX 4.1
%
% See the REVTeX 4 README file
% It also requires running BibTeX. The commands are as follows:
%
%  1)  latex apssamp.tex
%  2)  bibtex apssamp
%  3)  latex apssamp.tex
%  4)  latex apssamp.tex
%

\documentclass[%
 reprint,
%superscriptaddress,
%groupedaddress,
%unsortedaddress,
%runinaddress,
%frontmatterverbose, 
%preprint,
%showpacs,preprintnumbers,
%nofootinbib,
%nobibnotes,
%bibnotes,
 amsmath,amssymb,
 aps,
%pra,
%prb,
%rmp,
%prstab,
%prstper,
%floatfix,
]{revtex4-1}
\usepackage{ctex}

\usepackage{graphicx}% Include figure files
\usepackage{dcolumn}% Align table columns on decimal point
\usepackage{bm}% bold math

%\usepackage{hyperref}% add hypertext capabilities
%\usepackage[mathlines]{lineno}% Enable numbering of text and display math
%\linenumbers\relax % Commence numbering lines

%\usepackage[showframe,%Uncomment any one of the following lines to test 
%%scale=0.7, marginratio={1:1, 2:3}, ignoreall,% default settings
%%text={7in,10in},centering,
%%margin=1.5in,
%%total={6.5in,8.75in}, top=1.2in, left=0.9in, includefoot,
%%height=10in,a5paper,hmargin={3cm,0.8in},
%]{geometry}

\begin{document}

\preprint{APS/123-QED}

\title{$^{22}\textnormal{Al}$缓发质子衰变}% Force line breaks with \\
\thanks{题目脚注????}%

% \author[b,c,d]{L.J.Sun(孙立杰)}
\author[b,c]{X.X.Xu(徐新星)}   
\author[b,e]{C.J.Lin(林承键)}
\author[c]{J.Lee(李晓菁)}
\author[f]{S.Q.Hou(侯素青)}
\author[g]{C.X.Yuan(袁岑溪)}
\author[a]{Z.H.Li(李智焕) \thanks{Corresponding author: zhli@pku.edu.cn}}
\author[h,i]{J.Jos\'e}
\author[j,k]{J.J.He(何建军)}
\author[j]{J.S.Wang(王建松)}
\author[b]{D.X.Wang(王东玺)}
\author[a]{H.Y.Wu(吴鸿毅)}
\author[c]{P.F.Liang(梁鹏飞)}
\author[f]{Y.Y.Yang(杨彦云)}
\author[f]{Y.H.Lam(蓝乙华)}
\author[f]{P.Ma(马朋)}
\author[l,f]{F.F.Duan(段芳芳)}
\author[f,l]{Z.H.Gao(高志浩)}
\author[f]{Q.Hu(胡强)}
\author[f]{Z.Bai(白真)}
\author[f]{J.B.Ma(马军兵)}
\author[f]{J.G.Wang(王建国)}
\author[b,e]{F.P.Zhong(钟福鹏)}
\author[a]{C.G.Wu(武晨光)}
\author[a]{D.W.Luo(罗迪雯)}
\author[a]{Y.Jiang(蒋颖)}
\author[a]{Y.Liu(刘洋)}
\author[f]{D.S.Hou(侯东升)}
\author[f]{R.Li(李忍)}
\author{N.R.Ma(马南茹)}
\author[f,m]{W.H.Ma(马维虎)}
\author[f]{G.Z.Shi(石国柱)}
\author[f]{G.M.Yu(余功明)}
\author[f]{D.Patel}
\author[f]{S.Y.Jin(金树亚)}
\author[n,f]{Y.F.Wang(王煜峰)}
\author[n,f]{Y.C.Yu(余悦超)}
\author[o,f]{Q.W.Zhou(周清武)}
\author[o,f]{P.Wang(王鹏)}
\author[p]{L.Y.Hu(胡力元)}
\author[a]{X.Wang(王翔)}
\author[a]{H.L.Zang(臧宏亮)}
\author[c]{P.J.Li(李朋杰)}
\author[c]{Q.Q.Zhao(赵青青)}
\author[b]{L.Yang(杨磊)}
\author[b]{P.W.Wen(温培威)}
\author[b]{F.Yang(杨峰)}
\author[b]{H.M.Jia(贾会明)}
\author{G.L.Zhang(张高龙)}
\author[b]{M.Pan(潘敏)}
\author{X.Y.Wang(汪小雨)}
\author[b]{H.H.Sun(孙浩瀚)}
\author[f]{Z.G.Hu(胡正国)}
\author[f]{R.F.Chen(陈若富)}
\author[f]{M.L.Liu(刘敏良)}
\author[f]{W.Q.Yang(杨维青)}
\author[d]{Y.M.Zhao(赵玉民)}
\author[b]{H.Q.Zhang(张焕乔)}

\affil[a]{State Key Laboratory of Nuclear Physics and Technology,
School of Physics, Peking University, Beijing 100871, China}
\affil[b]{Department of Nuclear Physics, China Institute of Atomic Energy, Beijing 102413, China}
\affil[c]{Department of Physics, The University of Hong Kong, Hong Kong, China}
\affil[d]{School of Physics and Astronomy, Shanghai Jiao Tong University, Shanghai 200240, China}
\affil[e]{College of Physics and Technology, Guangxi Normal University, Guilin 541004, China}
\affil[f]{Institute of Modern Physics, Chinese Academy of Sciences, Lanzhou 730000, China}
\affil[g]{Sino-French Institute of Nuclear Engineering and Technology, Sun Yat-Sen University, Zhuhai 519082, China}
\affil[h]{Departament de F\'isica, EEBE, Universitat Polit\'ecnica de Catalunya, Av./ Eduard Maristany 10, E-08930 Barcelona, Spain}
\affil[i]{Institut d'Estudis Espacials de Catalunya (IEEC),
Ed. Nexus-201, C/ Gran Capit\'a 2-4, E-08034 Barcelona, Spain}
\affil[j]{Key Laboratory of Optical Astronomy, National Astronomical Observatories,
Chinese Academy of Sciences, Beijing 100012, China}
\affil[k]{University of Chinese Academy of Sciences, Beijing 100049, China}
\affil[l]{School of Nuclear Science and Technology, Lanzhou University, Lanzhou 730000, China}
\affil[m]{Institute of Modern Physics, Fudan University, Shanghai 200433, China}
\affil[n]{School of Physics and Astronomy, Yunnan University, Kunming 650091, China}
\affil[o]{School of Physical Science and Technology, Southwest University, Chongqing 400044, China}
\affil[p]{Fundamental Science on Nuclear Safety and Simulation Technology Laboratory,
Harbin Engineering University, Harbin 150001, China}
\affil[q]{School of Physics and Nuclear Energy Engineering, Beihang University, Beijing 100191, China}

% \collaboration{MUSO Collaboration}%\noaffiliation


% \collaboration{RIBLL Collaboration}%\noaffiliation

\date{\today}% It is always \today, today,
             %  but any date may be explicitly specified

\begin{abstract}
An article usually includes an abstract, a concise summary of the work
covered at length in the main body of the article. 
这里是中文测试。
\begin{description}
\item[Usage]
Secondary publications and information retrieval purposes.
\item[PACS numbers]
May be entered using the \verb+\pacs{#1}+ command.
\item[Structure]
You may use the \texttt{description} environment to structure your abstract;
use the optional argument of the \verb+\item+ command to give the category of each item. 
\end{description}
\end{abstract}

\pacs{Valid PACS appear here}% PACS, the Physics and Astronomy
                             % Classification Scheme.
%\keywords{Suggested keywords}%Use showkeys class option if keyword
                              %display desired
\maketitle

%\tableofcontents

\section{\label{sec:level1}First-level heading}

This sample document demonstrates proper use of REV\TeX~4.1 (and
\LaTeXe) in mansucripts prepared for submission to APS
journals. Further information can be found in the REV\TeX~4.1
documentation included in the distribution or available at
\url{http://authors.aps.org/revtex4/}.

When commands are referred to in this example file, they are always
shown with their required arguments, using normal \TeX{} format. In
this format, \verb+#1+, \verb+#2+, etc. stand for required
author-supplied arguments to commands. For example, in
\verb+\section{#1}+ the \verb+#1+ stands for the title text of the
author's section heading, and in \verb+\title{#1}+ the \verb+#1+
stands for the title text of the paper.

Line breaks in section headings at all levels can be introduced using
\textbackslash\textbackslash. A blank input line tells \TeX\ that the
paragraph has ended. Note that top-level section headings are
automatically uppercased. If a specific letter or word should appear in
lowercase instead, you must escape it using \verb+\lowercase{#1}+ as
in the word ``via'' above.

\subsection{\label{sec:level2}Second-level heading: Formatting}

This file may be formatted in either the \texttt{preprint} or
\texttt{reprint} style. \texttt{reprint} format mimics final journal output. 
Either format may be used for submission purposes. \texttt{letter} sized paper should
be used when submitting to APS journals.

\subsubsection{Wide text (A level-3 head)}
The \texttt{widetext} environment will make the text the width of the
full page, as on page~\pageref{eq:wideeq}. (Note the use the
\verb+\pageref{#1}+ command to refer to the page number.) 
\paragraph{Note (Fourth-level head is run in)}
The width-changing commands only take effect in two-column formatting. 
There is no effect if text is in a single column.

\subsection{\label{sec:citeref}Citations and References}
A citation in text uses the command \verb+\cite{#1}+ or
\verb+\onlinecite{#1}+ and refers to an entry in the bibliography. 
An entry in the bibliography is a reference to another document.

\subsubsection{Citations}
Because REV\TeX\ uses the \verb+natbib+ package of Patrick Daly, 
the entire repertoire of commands in that package are available for your document;
see the \verb+natbib+ documentation for further details. Please note that
REV\TeX\ requires version 8.31a or later of \verb+natbib+.

\paragraph{Syntax}
The argument of \verb+\cite+ may be a single \emph{key}, 
or may consist of a comma-separated list of keys.
The citation \emph{key} may contain 
letters, numbers, the dash (-) character, or the period (.) character. 
New with natbib 8.3 is an extension to the syntax that allows for 
a star (*) form and two optional arguments on the citation key itself.
The syntax of the \verb+\cite+ command is thus (informally stated)
\begin{quotation}\flushleft\leftskip1em
\verb+\cite+ \verb+{+ \emph{key} \verb+}+, or\\
\verb+\cite+ \verb+{+ \emph{optarg+key} \verb+}+, or\\
\verb+\cite+ \verb+{+ \emph{optarg+key} \verb+,+ \emph{optarg+key}\ldots \verb+}+,
\end{quotation}\noindent
where \emph{optarg+key} signifies 
\begin{quotation}\flushleft\leftskip1em
\emph{key}, or\\
\texttt{*}\emph{key}, or\\
\texttt{[}\emph{pre}\texttt{]}\emph{key}, or\\
\texttt{[}\emph{pre}\texttt{]}\texttt{[}\emph{post}\texttt{]}\emph{key}, or even\\
\texttt{*}\texttt{[}\emph{pre}\texttt{]}\texttt{[}\emph{post}\texttt{]}\emph{key}.
\end{quotation}\noindent
where \emph{pre} and \emph{post} is whatever text you wish to place 
at the beginning and end, respectively, of the bibliographic reference
(see Ref.~[\onlinecite{witten2001}] and the two under Ref.~[\onlinecite{feyn54}]).
(Keep in mind that no automatic space or punctuation is applied.)
It is highly recommended that you put the entire \emph{pre} or \emph{post} portion 
within its own set of braces, for example: 
\verb+\cite+ \verb+{+ \texttt{[} \verb+{+\emph{text}\verb+}+\texttt{]}\emph{key}\verb+}+.
The extra set of braces will keep \LaTeX\ out of trouble if your \emph{text} contains the comma (,) character.

The star (*) modifier to the \emph{key} signifies that the reference is to be 
merged with the previous reference into a single bibliographic entry, 
a common idiom in APS and AIP articles (see below, Ref.~[\onlinecite{epr}]). 
When references are merged in this way, they are separated by a semicolon instead of 
the period (full stop) that would otherwise appear.

\paragraph{Eliding repeated information}
When a reference is merged, some of its fields may be elided: for example, 
when the author matches that of the previous reference, it is omitted. 
If both author and journal match, both are omitted.
If the journal matches, but the author does not, the journal is replaced by \emph{ibid.},
as exemplified by Ref.~[\onlinecite{epr}]. 
These rules embody common editorial practice in APS and AIP journals and will only
be in effect if the markup features of the APS and AIP Bib\TeX\ styles is employed.

\paragraph{The options of the cite command itself}
Please note that optional arguments to the \emph{key} change the reference in the bibliography, 
not the citation in the body of the document. 
For the latter, use the optional arguments of the \verb+\cite+ command itself:
\verb+\cite+ \texttt{*}\allowbreak
\texttt{[}\emph{pre-cite}\texttt{]}\allowbreak
\texttt{[}\emph{post-cite}\texttt{]}\allowbreak
\verb+{+\emph{key-list}\verb+}+.

\end{document}
%
% ****** End of file apssamp.tex ******
