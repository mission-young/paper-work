\documentclass[UTF8]{ctexart}
\usepackage{xcolor}
% \usepackage{amsmath}
\begin{document}
\title{题目}
\thanks{脚注}
\author{作者}
\author{二作}   
\author{三作者}
\date{\today}

\begin{abstract}
    \textbf{Abstract:}
    \newline
    {\color{red}?} 核的 {\color{red}$\beta$衰变} 属性在 {\color{red}实验地点}通过 {\color{red}带电粒子和 $\gamma$谱学} 方式被调研.
    \begin{enumerate}
        \item The $\beta$-decay properties of the neutron-deficient nuclei {\color{red}$^{25}$Si and $^{26}$P} {\color{blue}$^{22}$Al} have been investigated at {\color{red}the GANIL/LISE3 facility} {\color{blue}Radioactive Ion Beam Line in Lanzhou (RIBLL)} by means of charged-particle and $\gamma$-ray spectroscopy.
    \end{enumerate}
    {\color{red}能级纲图和跃迁强度核心的实验和理论对比描述。}
    \begin{enumerate}
        \item  The decay schemes obtained and the Garrow-Teller strength distributions are compared to shell-model calculations based on the USD interaction. 
    \end{enumerate}
    {\color{red}实验结果给出的一些信息}
    \begin{enumerate}
        \item $B(GT)$ values derived from the absolute measurement of the $\beta$-decay braching ratios give rise to a quenching factor of the Gamow-Teller strength of {\color{red}0.6}{\color{blue}null}. 
    \end{enumerate}
    {\color{red} 半衰期和衰变模式}
    \begin{enumerate}
        \item A precise half-life of {\color{red}43.7(6)}{\color{blue}null} ms was detemined for {\color{red}$^{26}$P}{\color{blue}$^{22}$Al},the $\beta$-(2)$p$ decay mode of which is described.
    \end{enumerate}
    \begin{description}
        \item[Usage] usage
        \item[PACS number] number
        \item[structure]   struct
    \end{description}

\end{abstract}
\maketitle

\section{introduction}
\subsection{generalities}
$\star\star\star\star ~background~description~ \star\star\star\star$

过去很多年, {\color{red}?}核的 {\color{red}?}属性已经被调研以验证??
\begin{enumerate}
    \item Over the last decades, $\beta$-decay properties of light unstable nuclei have been extensively investigated in order to probe their single-particle nuclear structure and to establish the proton and neutron drip lines.
\end{enumerate}

$\star\star\star\star ~shell ~model~  \star\star\star\star$

在非稳定核区域,$\beta$衰变可以用来检验这些模型。

Hence, compilation of spectroscopic properites are available for many $sd$ shell nuclei from which nucleon-nucleon interactions were derived. $\beta$-decay studies of nuclei having a large proton excess are therefore useful to test the validity of these models when they are applied to very unstable nuclei.

$\star\star\star\star ~reduced ~transition ~probability ~ft \star\star\star\star$

约化跃迁强度ft用于描述实验结果和弱相互作用基本参数的物理量。

Moreover, in the standard V-A description of $\beta$-decay, a direct link between experiment results and fundamental constants of the weak interaction is given by the reduced transition probability $ft$ of the individual allowed $\beta$-decays. This parameter,which incorporates the phase space factor $f$ and the parital half-life $t=T_{1/2}/\rm{B.R}$. ($T_{1/2}$ being the total half-life of the decaying nucleus and B.R. the branching ratio associated with the $\beta$ transition considered), can be written as follow:

$\star\star\star\star~equation~ft~\star\star\star\star$

\begin{equation}
    ft=\frac{\kappa}{g^{2}_{V}\Big|\langle f|\tau|i \rangle\Big|^{2}+g^{2}_{A}\Big|\langle f|\sigma\tau|i \rangle\Big|^{2}}
\end{equation}

$\star\star\star\star~ft~explanation~\star\star\star\star$

where $\kappa$ is a constant and where $g_{V}$ and $g_{A}$ are, respectively, the vector and axial-vector current coupling constants related to the Fermi and Gamow-Teller components of $\beta$-decay. $\tau$ and $\sigma$ are the isospin and the spin operators, respectively. 

$\star\star\star\star~ft~meanings~\star\star\star\star$

测量所得的ft值与Fermi和GT计算所得的比较,用于检验壳模型波函数,可以看出原子核初末态波函数重叠以及父态、子态的组态混合。

Hence, the comparison of the measured $ft$ values and the computed Fermi and Gamow-Teller matrix elements appears to be a good test of nuclear wave fucntions built in the shell-model frame, stressing the role of the overlap between initial and final nuclear states as well as the configuration mixing occurring in parent and daughter states. 

$\star\star\star\star~mirror~asymmetry~anomaly~and~quenching~of~the~GT~strength~\star\star\star\star$

两处系统性地偏离理论预测的情形体现了理论理解和基础相互作用认知的局限,即 $\beta$衰变同位旋对称性破缺以及GT强度的猝灭。

However, two systematic deviations from theoretical predictions shows the limitation of our theoretical understanding and treatment of fundamental interactions. They are reported as the \textit{mirror asymmetry anomaly in $\beta$} and the \textit{quenching of the Gamow-Teller strength}.

$\star\star\star\star~Mirror~asymmetry~in~\beta~decay~\star\star\star\star$

$\beta$衰变同位旋破缺的描述和理解。

\textit{Mirror asymmetry in $\beta$-decay}: This phenomenon is related to the isospin non-conserving forces acting in the atomic nucleus. If nuclear forces were charge independent, the $\beta^{+}$(EC) and $\beta^{-}$ decays of analog states belonging to mirror nuclei would be of equal strength. The deviation from this simple picture is characterized by the asymmetry parameter $\delta=\left( ft^{+}/ft^{-}-1 \right)$, where the + and - signs are associated with the decay of the proton- and the neutron- rich members of the mirror pair, respectively. Figure 1 presents an updated systematic of $\delta$ values measured for mirror nuclei with A $\leq$ 40. Thirty-nine allowed Gamow-Teller mirror transitions with log(\textit{ft}) $\leq$ 6 pertaining to 14 pairs of a mean deviation of about 5\% for these nuclei lying in the $p$ and $sd$ shells. The asymmetry reaches 11(1)\% if only $p$ shell nuclei are considered, which stresses the interplay between the Coulomb and the centrifugal barriers.

对称性破缺的两种理论解释。次级电流和波函数重叠。

It was often attempted to explain the mirror asymmetry anormaly in the $p$ shell either in terms of binding effects or by troducing the concept of " second-class currents ", which are not allowed within the frame of the standard V-A model of the weak interaction. None of the theoretical approaches ware able to reproduced the measured $\delta$ values. Shell-model calculations are currently performed to test the isospin non-conserving part of the interaciton in $\beta$-decay by studying the influence of isospin mixing effects and of radial overlap mismatches of nuclear wave functions on the Gamow-Teller matrix elements. These calculations are performed in the $p$ shell and in the $sd$ shell, where reliable single-particle nuclear wave functions are now available.


$\star\star\star\star~Gamow-Teller~quchching~\star\star\star\star$



\textit{Gamow-Teller quenching:} The axial-vector couping constant $g_{A}$ involved in $\beta$ transitions of the Gamow-Teller type is not strictly constant and it has to be renormalized in order to reproduce the $ft$ values measured experimentally. The effective couping constant $g_{A,eff}=q\cdot g_{A}$ is deduced empirically from nuclear-structure experiments and shows a slight variation over a wide range of masses: $q=0.820(15)$ in the $p$ shell,$q=0.77(2)$ in the $sd$ shell(giving a quenching factor $q^{2}$ of 0.6) and $q=0.744(15)$ in the $pf$ shell.

Different theoretical approaches haves been used in order to derive the renormalization factor from core polarization effects (due to particle-hole excitations), isobar currents and meson exchange. Despite all these efforts, the origin of the quenching effect is not very well understood. Nevertheless, the Gamow-Teller strength function $B(GT)=\left( g_{A}/g_{V} \right)^{2}\left| \sigma \tau  \right|^{2}$, which translates the global response of the wave function to spin-isospin excitations ocuurring in the $\beta$-decay,is a useful link between experimental results and theoretical predictions and it can be used as a comparative tool.

$\star\star\star\star~Experimental~development~\star\star\star\star$



\textit{Experimental development:} With the development of secondary radioactive beams and other experimental techiques like the combination of helium-jet transport systems with telescope detectors, a large set of neutron-deficient nuclei has been investigated since the $\beta$-decayed proton emission was first first observed forty years ago. As $Q_{EC}$ values are increasing while nuclei become more exotic, $\beta-p$ and $\beta-\gamma$ spectroscopic studies of neutron-deficient nuclei give the opportunity to probe the Gamow-Teller strength function up to more than 10 MeV in excitation energy. Hence, the whole energy window open in $\beta$-decay can be covered both by spectroscopic studies and charge exchange reactions. Therefore, the theoretical description of nuclear structure as well as our understanding of the weak interaction can be tested far from the stability line. As an illustration,we will report in the following on the $\beta$-decay properties of two neutron-deficient light nuclei, namely $^{25}\textnormal{Si}$ and $^{26}\textnormal{P}$.

\subsection{Previous studies}
\subsubsection{Studies of $^{25}\textnormal{Si}$}
描述先前测得核的半衰期,$Q_{\textnormal{EC}}$,同位旋等信息。
\begin{enumerate}
    \item With a lifetime of 218 ms and a $Q_{\textnormal{EC}}$ value of about 13 MeV, the $T_{Z}=-\frac{3}{2}$ nucleus $^{25}\textnormal{Si}$ has been studies several times since the end of the 1960s. 
\end{enumerate}

描述先前实验在本次实验中承担的作用,分支比对比,刻度等等。
\begin{enumerate}
    \item These previous studies will be used in the present work to validate the analysis procedure implemented to derive the $\beta$-decay properties of $^{26}\textnormal{P}$. 
\end{enumerate}

描述先前实验中不足的地方。如缺乏某类灵敏探测器。探测器分辨不足,立体角覆盖小等等。
\begin{enumerate}
    \item However, none of these studies measured the decay by $\gamma$ emission of excited stated fed in the $\beta$-decay of $^{25}\textnormal{Si}$.
\end{enumerate}



\subsubsection{Studies of $^{26}\textnormal{P}$}
\subsubsection{Present measurement}
\section{Experiment procedure}
\subsection{Fragment production and detection set-up}
\subsection{$\beta$-delayed proton spectroscopy}
\subsubsection{Energy calibration of the implantation detector}
\subsubsection{Proton detection efficiency}
\subsubsection{Absolute intensities of the observed proton groups}
\subsection{$\gamma$-ray spectroscopy}
\section{Experimental results}
\subsection{$\beta$-decay study of $^{25}\textnormal{Si}$}
\subsubsection{$\beta$-decay proton emission}
\subsubsection{$\beta$-delayed $\gamma$-decay}
\subsubsection{$\beta$-decay scheme of $^{25}\textnormal{Si}$}
\subsection{$\beta$-decay study of $^{26}\textnormal{P}$}
\subsubsection{$\beta$-delayed proton emission}
\subsubsection{$\beta$-delayed $\gamma$-decay}
\subsubsection{Measurement of the half-life of $^{26}P$}
\subsubsection{$\beta$-decay scheme of $^{26}\textnormal{P}$}
\subsection{Mirror asymmetry of mass A=25,26 nuclei}
\section{Conclusion and perspectives}

\end{document}