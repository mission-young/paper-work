\documentclass[UTF8]{ctexart}
\usepackage{xcolor}
% \usepackage{amsmath}
\begin{document}
\title{题目}
\thanks{脚注}
\author{作者}
\author{二作}   
\author{三作者}
\date{\today}

\begin{abstract}
    \textbf{Abstract:}
    \newline
    {\color{red}?} 核的 {\color{red}$\beta$衰变} 属性在 {\color{red}实验地点}通过 {\color{red}带电粒子和 $\gamma$谱学} 方式被调研.
    \begin{enumerate}
        \item The $\beta$-decay properties of the neutron-deficient nuclei {\color{red}$^{25}$Si and $^{26}$P} {\color{blue}$^{22}$Al} have been investigated at {\color{red}the GANIL/LISE3 facility} {\color{blue}Radioactive Ion Beam Line in Lanzhou (RIBLL)} by means of charged-particle and $\gamma$-ray spectroscopy.
    \end{enumerate}
    {\color{red}能级纲图和跃迁强度核心的实验和理论对比描述。}
    \begin{enumerate}
        \item  The decay schemes obtained and the Garrow-Teller strength distributions are compared to shell-model calculations based on the USD interaction. 
    \end{enumerate}
    {\color{red}实验结果给出的一些信息}
    \begin{enumerate}
        \item $B(GT)$ values derived from the absolute measurement of the $\beta$-decay braching ratios give rise to a quenching factor of the Gamow-Teller strength of {\color{red}0.6}{\color{blue}null}. 
    \end{enumerate}
    {\color{red} 半衰期和衰变模式}
    \begin{enumerate}
        \item A precise half-life of {\color{red}43.7(6)}{\color{blue}null} ms was detemined for {\color{red}$^{26}$P}{\color{blue}$^{22}$Al},the $\beta$-(2)$p$ decay mode of which is described.
    \end{enumerate}
    \begin{description}
        \item[Usage] usage
        \item[PACS number] number
        \item[structure]   struct
    \end{description}

\end{abstract}
\maketitle

\section{introduction}
\subsection{generalities}
$\star\star\star\star ~background~description~ \star\star\star\star$

过去很多年, {\color{red}?}核的 {\color{red}?}属性已经被调研以验证??
\begin{enumerate}
    \item Over the last decades, $\beta$-decay properties of light unstable nuclei have been extensively investigated in order to probe their single-particle nuclear structure and to establish the proton and neutron drip lines.
\end{enumerate}

$\star\star\star\star ~shell ~model~  \star\star\star\star$

在非稳定核区域,$\beta$衰变可以用来检验这些模型。

Hence, compilation of spectroscopic properites are available for many $sd$ shell nuclei from which nucleon-nucleon interactions were derived. $\beta$-decay studies of nuclei having a large proton excess are therefore useful to test the validity of these models when they are applied to very unstable nuclei.

$\star\star\star\star ~reduced ~transition ~probability ~ft \star\star\star\star$

约化跃迁强度ft用于描述实验结果和弱相互作用基本参数的物理量。

Moreover, in the standard V-A description of $\beta$-decay, a direct link between experiment results and fundamental constants of the weak interaction is given by the reduced transition probability $ft$ of the individual allowed $\beta$-decays. This parameter,which incorporates the phase space factor $f$ and the parital half-life $t=T_{1/2}/\rm{B.R}$. ($T_{1/2}$ being the total half-life of the decaying nucleus and B.R. the branching ratio associated with the $\beta$ transition considered), can be written as follow:

$\star\star\star\star~equation~ft~\star\star\star\star$

\begin{equation}
    ft=\frac{\kappa}{g^{2}_{V}\Big|\langle f|\tau|i \rangle\Big|^{2}+g^{2}_{A}\Big|\langle f|\sigma\tau|i \rangle\Big|^{2}}
\end{equation}

$\star\star\star\star~ft~explanation~\star\star\star\star$

where $\kappa$ is a constant and where $g_{V}$ and $g_{A}$ are, respectively, the vector and axial-vector current coupling constants related to the Fermi and Gamow-Teller components of $\beta$-decay. $\tau$ and $\sigma$ are the isospin and the spin operators, respectively. 

$\star\star\star\star~ft~meanings~\star\star\star\star$

测量所得的ft值与Fermi和GT计算所得的比较,用于检验壳模型波函数,可以看出原子核初末态波函数重叠以及父态、子态的组态混合。

Hence, the comparison of the measured $ft$ values and the computed Fermi and Gamow-Teller matrix elements appears to be a good test of nuclear wave fucntions built in the shell-model frame, stressing the role of the overlap between initial and final nuclear states as well as the configuration mixing occurring in parent and daughter states. 

$\star\star\star\star~mirror~asymmetry~anomaly~and~quenching~of~the~GT~strength~\star\star\star\star$

两处系统性地偏离理论预测的情形体现了理论理解和基础相互作用认知的局限,即 $\beta$衰变同位旋对称性破缺以及GT强度的猝灭。

However, two systematic deviations from theoretical predictions shows the limitation of our theoretical understanding and treatment of fundamental interactions. They are reported as the \textit{mirror asymmetry anomaly in $\beta$} and the \textit{quenching of the Gamow-Teller strength}.

$\star\star\star\star~Mirror~asymmetry~in~\beta~decay~\star\star\star\star$

$\beta$衰变同位旋破缺的描述和理解。

\textit{Mirror asymmetry in $\beta$-decay}: This phenomenon is related to the isospin non-conserving forces acting in the atomic nucleus. If nuclear forces were charge independent, the $\beta^{+}$(EC) and $\beta^{-}$ decays of analog states belonging to mirror nuclei would be of equal strength. The deviation from this simple picture is characterized by the asymmetry parameter $\delta=\left( ft^{+}/ft^{-}-1 \right)$, where the + and - signs are associated with the decay of the proton- and the neutron- rich members of the mirror pair, respectively. Figure 1 presents an updated systematic of $\delta$ values measured for mirror nuclei with A $\leq$ 40. Thirty-nine allowed Gamow-Teller mirror transitions with log(\textit{ft}) $\leq$ 6 pertaining to 14 pairs of a mean deviation of about 5\% for these nuclei lying in the $p$ and $sd$ shells. The asymmetry reaches 11(1)\% if only $p$ shell nuclei are considered, which stresses the interplay between the Coulomb and the centrifugal barriers.

对称性破缺的两种理论解释。次级电流和波函数重叠。

It was often attempted to explain the mirror asymmetry anormaly in the $p$ shell either in terms of binding effects or by troducing the concept of " second-class currents ", which are not allowed within the frame of the standard V-A model of the weak interaction. None of the theoretical approaches ware able to reproduced the measured $\delta$ values. Shell-model calculations are currently performed to test the isospin non-conserving part of the interaciton in $\beta$-decay by studying the influence of isospin mixing effects and of radial overlap mismatches of nuclear wave functions on the Gamow-Teller matrix elements. These calculations are performed in the $p$ shell and in the $sd$ shell, where reliable single-particle nuclear wave functions are now available.


$\star\star\star\star~Gamow-Teller~quchching~\star\star\star\star$



\textit{Gamow-Teller quenching:} The axial-vector couping constant $g_{A}$ involved in $\beta$ transitions of the Gamow-Teller type is not strictly constant and it has to be renormalized in order to reproduce the $ft$ values measured experimentally. The effective couping constant $g_{A,eff}=q\cdot g_{A}$ is deduced empirically from nuclear-structure experiments and shows a slight variation over a wide range of masses: $q=0.820(15)$ in the $p$ shell,$q=0.77(2)$ in the $sd$ shell(giving a quenching factor $q^{2}$ of 0.6) and $q=0.744(15)$ in the $pf$ shell.

Different theoretical approaches haves been used in order to derive the renormalization factor from core polarization effects (due to particle-hole excitations), isobar currents and meson exchange. Despite all these efforts, the origin of the quenching effect is not very well understood. Nevertheless, the Gamow-Teller strength function $B(GT)=\left( g_{A}/g_{V} \right)^{2}\left| \sigma \tau  \right|^{2}$, which translates the global response of the wave function to spin-isospin excitations ocuurring in the $\beta$-decay,is a useful link between experimental results and theoretical predictions and it can be used as a comparative tool.

$\star\star\star\star~Experimental~development~\star\star\star\star$



\textit{Experimental development:} With the development of secondary radioactive beams and other experimental techiques like the combination of helium-jet transport systems with telescope detectors, a large set of neutron-deficient nuclei has been investigated since the $\beta$-decayed proton emission was first first observed forty years ago. As $Q_{EC}$ values are increasing while nuclei become more exotic, $\beta-p$ and $\beta-\gamma$ spectroscopic studies of neutron-deficient nuclei give the opportunity to probe the Gamow-Teller strength function up to more than 10 MeV in excitation energy. Hence, the whole energy window open in $\beta$-decay can be covered both by spectroscopic studies and charge exchange reactions. Therefore, the theoretical description of nuclear structure as well as our understanding of the weak interaction can be tested far from the stability line. As an illustration,we will report in the following on the $\beta$-decay properties of two neutron-deficient light nuclei, namely $^{25}\textnormal{Si}$ and $^{26}\textnormal{P}$.

\subsection{Previous studies}
\subsubsection{Studies of $^{25}\textnormal{Si}$}
描述先前测得核的半衰期,$Q_{\textnormal{EC}}$,同位旋等信息。
\begin{enumerate}
    \item With a lifetime of 218 ms and a $Q_{\textnormal{EC}}$ value of about 13 MeV, the $T_{Z}=-\frac{3}{2}$ nucleus $^{25}\textnormal{Si}$ has been studies several times since the end of the 1960s. 
\end{enumerate}

描述先前实验在本次实验中承担的作用,分支比对比,刻度等等。
\begin{enumerate}
    \item These previous studies will be used in the present work to validate the analysis procedure implemented to derive the $\beta$-decay properties of $^{26}\textnormal{P}$. 
\end{enumerate}

描述先前实验中不足的地方。如缺乏某类灵敏探测器。探测器分辨不足,立体角覆盖小等等。
\begin{enumerate}
    \item However, none of these studies measured the decay by $\gamma$ emission of excited stated fed in the $\beta$-decay of $^{25}\textnormal{Si}$.
\end{enumerate}

描述下最近的一个实验做的事情。

\begin{enumerate}
    \item The most recent $\beta$-delayed proton emission study of $^{25}\textnormal{Si}$ was performed by Robertson \textit{et al.} It updates the first investigation of Reeder \textit{et al.} in 1966. In both experiments, the individual proton group intensities were measured relative to the most intense one, emitted by the isobaric analog state (IAS) in $^{25}\textnormal{Al}$. The absolute $\beta$-decay branching ratio of 12.2\% towards this state was derived from the associated $\textnormal{log}\left( ft \right)$ value $\left( \textnormal{log}ft=3.28 \right)$, calculated assuming a pure Fermi $\beta$ transition from the ground state of $^{25}\textnormal{Si}$. It led to a summed $\beta$ feeding of proton-unbound states of $^{25}\textnormal{Al}$ equal to 28.1(15)\%. This normalization procedure is supported by the measurement of Hatori \textit{et al.}. 
\end{enumerate}

本工作和上述工作的对比,是否一致。

\begin{enumerate}
    \item In this work, absolute branching ratios for $\beta$-decay were determined by counting the total number of $\beta$-particles emitted with the half-life of $^{25}\textnormal{Si}$ and the $\beta$ feeding of the IAS in $^{25}\textnormal{Al}$ was indeed found to be equal to 14.6(6)\%, giving raise to a $\textnormal{log}ft$ value of 3.19(2). The summed feeding of the $^{25}\textnormal{Al}$ proton-emitting states was measured to be 40.7(14)\%, in good agreement with Robertson \textit{et al.}
\end{enumerate}

总结一下工作中的不足,以及可能带来的后果。
\begin{enumerate}
    \item As mentioned above, in none of the experiments, the $\beta$-delayed $\gamma$-decay of $^{25}\textnormal{Si}$ was observed. As a consequence, the $\beta$-decay branching ratios towards the proton-bound states of $^{25}\textnormal{Al}$ were tentatively estimated taking into account the summed $\beta$ feeding and assuming that the relative \textit{ft} values of these states were equal to those of the mirror states in $^{25}\textnormal{Mg}$. The weak point of such a procedure is that an average $\beta$ asymmetry of 20\% had to be taken into account for all proton-bound states, which was assumed to be equally shared by the proton-bound states disregarding their individual quantum characteristics.
\end{enumerate}

\subsubsection{Studies of $^{26}\textnormal{P}$}

Due to its $T_{Z}$ value of -2 and its short lifetime of less than 100ms, $^{26}\textnormal{P}$ has not been investigated in detail so far. Compilations only report the observation by Cable \textit{et al.} of $\beta$-delayed proton and two-proton emission from this nucleus. A half-life of $20_{-15}^{+35} \textnormal{ms}$ was deduced from the observation of the most intense proton group. It led to a $\beta$ feeding of the IAS in $^{26}\textnormal{Si}$ equal to $1.9_{-3.5}^{-1.4}\%$ using a calculated $\textnormal{log}\left( ft \right)$ value of 3.19 (assuming a pure Fermi transition). Only three proton groups were observed linking the IAS to the two lowest states of $^{25}\textnormal{Al}$ ($\beta-p$ decay) and to the ground state of $^{24}\textnormal{Mg}$ ($\beta-2p$ decay). The two decay modes of the IAS were reported to be of similar magnitude. However, the large $Q_{\textnormal{EC}}$ value of 18 MeV together with a proton speration energy of 5.5 MeV for the daughter nucleus $^{26}\textnormal{Si}$ are indications that the $\beta$-delayed charged-particle spectrum may be rather complex, involving a large number of proton groups.

\subsubsection{Present measurement}

在这个实验中,通过??方式确定了绝对分支比。
\begin{enumerate}
    \item In our experiment, we determined the absolute branching ratios for $^{25}\textnormal{Si}$ and $^{26}\textnormal{P}$ by relating the intensity of a given proton or $\gamma$ line to the number of isotopes of each type implanted in our set-up. 
\end{enumerate}
For $^{25}\textnormal{Si}$, this measurement consititude a first unambigous determination of braching ratios also for proton-bound levels. We will use the decay of $^{25}\textnormal{Si}$ in part to test our analysis procedure, however, our study yields also new results for this nucleus, in particular for the $\gamma$-decay of its $\beta$-decay daughter. In the case of $^{26}\textnormal{P}$, we deduce for the first time the feeding for other states than the IAS and their decay by proton or $\gamma$ emission. Therefore, we could establish a complete decay scheme for braches with more than about 1\% feeding for both nuclei for the first time.
\section{Experiment procedure}
\subsection{Fragment production and detection set-up}

描述除了目标核外还有哪些核也被产生,以及他们各自的目的。
\begin{enumerate}
    \item In additon to $^{25}\textnormal{Si}$ and $^{26}\textnormal{P}$, the $\beta$-delayed proton and two-proton emitters $^{22}\textnormal{Al}$ and $^{27}\textnormal{S}$ have been studied during the same experimental campaign. The $\beta$-delayed proton emitter $^{21}\textnormal{Mg}$ and the $\beta$-delayed $\gamma$ emitter $^{24}\textnormal{Al}$ were also produced for calibration and efficiency measurement purposes.
\end{enumerate}

描述束流强度,种类以及设施。
\begin{enumerate}
    \item All nuclei have been produced in the fragmentation of a 95 MeV / \textit{u} $^{26}\textnormal{Ar}^{18+}$ primary beam with an intensity of about 2 $\mu$A$e$ delivered by the coupled cyclotrons of the GANIL facility.
\end{enumerate}

描述靶材设置以及其他束流选取的设置。
\begin{enumerate}
    \item A 357.1mg/cm$^{2}$ $^{12}\textnormal{C}$ production target was placed in the SISSI device, the high angular acceptance and focusing properites of which increased the selectivity of the fragment separation operated by the LISE3 spectrometer. The latter included a shaped Be degrader (thickness 1062 $\mu$m) at the intermediate focal plane and a Wien filter at the end of the line to refine the selection of the separated fragments.
\end{enumerate}

感兴趣核注入位置描述。
\begin{enumerate}
    \item Ions of interest were implanted in the fourth element \textit{E}4 of a silicon stack.
\end{enumerate}

粒子鉴别方式。
\begin{enumerate}
    \item The ion identification was performed by means of time-of-flight and energy loss measurements with the silicon detecotrs \textit{E}1 to \textit{E}4 (2*300 $\mu$m and 2*500 $\mu$m in thickness, 4*600 mm$^{2}$ of surface).
\end{enumerate}

 It led to a precision in the counting rate of better than 1\% for $^{25}\textnormal{Si}$ and about 3\% for the more exotic $^{26}\textnormal{P}$ nucleus. The production method in association with the high selectivity of the LISE3 spectrometer gave rise to a very low contamination rate of the selected species by only a few isotones.

探测器之间的符合关联
\begin{enumerate}
    \item Protons were detected in the implantation detector \textit{E}4, in coincidence with the observation of $\beta$-particles in the detector \textit{E}5 (with a thickness of 6 mm and an area of 600 mm$^{2}$). A segmented germanium clover was finally used to study the $\beta$-delayed $\gamma$-decay of implanted ions.
\end{enumerate}

\subsection{$\beta$-delayed proton spectroscopy}
$\beta$叠加效应部分的描述
\begin{enumerate}
    \item Contrary to previous experiments in which ions were deposited at the surface of an ion cather, $\beta$-delayed protons are emitted inside the implantation detecotr \textit{E}4. As a first consequence, the proton spectrum rises on a large $\beta$ background and the identification of low-energy, low-intensity proton lines is difficult. Secondly, the energy deposit in the detector \textit{E}4 of an emitted proton cannot be disentangled from the energy loss contribution of the associated $\beta$-particle and the recoiling ion.
\end{enumerate}

减少 $\beta$叠加效应
\begin{enumerate}
    \item To minimize these effects, ions were implanted in the last 100 $\mu$m of the detector \textit{E}4 and a $\beta$ coincidence with the thicker detector \textit{E}5 was requested in the analysis. As shown in the upper part of fig.3, the $\beta$-particle energy deposit in the coincidence spectrum was strongly reduced and proton peaks could be easily identified and fitted with the help of Gaussian distributions. The energy calibration of the detector \textit{E}4 as well as the measurement of the proton group intensities were performed on the basis of this \textit{E}4-\textit{E}5 coincidence condition.
\end{enumerate}

\subsubsection{Energy calibration of the implantation detector}
$\beta$叠加效应正比于注入深度
\begin{enumerate}
    \item The $\beta$-particle energy deposit leads to a shift in energy of the Gaussian-like part of the proton peaks. This effect could be reproduced by means of a GEANT simulation, as shown in the lower part of fig.3 for a representative $\beta$-delayed proton peak. It could also be shown that the energy shift is roughly independent of the proton and $\beta$-particle energies but linearly dependent on the implantation depth of the ions, that is to say, on the distance the $\beta$-particles travel in the detector \textit{E}4 before leaving it to enter the coincidence detector \textit{E}5. The energy calibrations of the detector \textit{E}4 for the settings on $^{21}\textnormal{Mg}$,$^{25}\textnormal{Si}$ and $^{26}\textnormal{P}$ were therefore assumed to differ only by a shift proportional to the implantation depths of the ions.
\end{enumerate}

使用$^{25}\textnormal{Si}$的一系列质子峰来进行刻度。
\begin{enumerate}
    \item The calibration parameters for the settings were deduced from the identification of the major proton groups expected at 1315(9),1893(2),2037(4),2589(9),4908(3) and 6542(3) keV for the decay of $^{21}\textnormal{Mg}$ and at 402(1),1925(3),2169(7),2312(4),3472(10),4261(2) and 5630(2) keV for the decay of $^{25}\textnormal{Si}$. The proton group energies were recalculated using the excitation energies of the proton-emitting states and the proton separation energies reported in a compilation.
\end{enumerate}

\subsubsection{Proton detection efficiency}
描述影响探测效率的因素。以及Geant4模拟的效果。

\begin{enumerate}
    \item Since ions were implanted at the end of the detector \textit{E}4, the proton detection efficiency $\varepsilon_{p}$ is very sensitive to the implantation profile of the emitting ion and to the proton energy. The detection efficiency for protons between 0.5 and 10 MeV was computed by means of GEATN4 simulations. Following experimental observations, implantation profiles were approximated by Gaussian distributions in the beam direction (with a standard deviation of 20 $\mu$m) and with a two-dimensional square-shaped function in the orthogonal plane. 
    
    Results are shown in fig.4. An uncertainty on the detection efficiency of less than 6\% was obtained. This uncertainty was determined by varying the implantation depth by $\pm$10 $\mu$m, which is roughly the width of the implantation distribution.
\end{enumerate}

\subsubsection{Absolute intensities of the observed proton groups}
质子峰强度公式
\begin{enumerate}
    \item The absolute intensity $I_{p}^{i}$ of a given proton group \textit{i} was derived from the following relation:
    \begin{equation}
        I_{p}^{i}=\frac{S_{c_{p}^{i}}}{K_{c_{p}} \cdot N_{\textnormal{impl}}\cdot \varepsilon_{p}^{i}}
    \end{equation}
\end{enumerate}

公式描述
\begin{enumerate}
    \item where $S_{c_{p}^{i}}$ is the area of the proton peak observed in the coincidence spectrum (\textit{E}5>0),$K_{c_{p}}$ the normalization factor to be taken into account due to the coincidence condition, $N_{\textnormal{impl}}$ the number of ions implanted in \textit{E}4 and $\varepsilon_{p}^{i}$ the proton detection efficiency for a given proton energy.
\end{enumerate}

相关参数的获得方式以及提取质子峰面积的方式
\begin{enumerate}
    \item The extraction of the factor $K_{c_{p}}$ is illustrated in fig.5 for the setting on $^{25}\textnormal{Si}$. Several proton peaks were fitted in the high-energy part of the \textit{E}4 energy spectrum, where the $\beta$ background is low enough and where proton peaks are well separated. The $K_{c_{p}}$ coefficients were deduced from the average ratio of the areas of the $\beta$-delayed proton peaks obtained with and without coincidence condition. For the coincidence spectrum, peaks were fitted by means of Gaussian distributions on a linear background (see inserts of fig.5) leading to the $S_{c_{p}^{i}}$ values. For the unconditioned energy spectrum, fit fucntios convoluting a Gaussian distribution and an exponential tail on top of an exponential background were used. For each ion of interest, the parameters of the exponential tail were fixed regardless of the proton peak energies. The $K_{c_{p}}$ coefficients obtained were about 13\%, with an uncertainty of 1 to 2\%.
\end{enumerate}


\subsection{$\gamma$-ray spectroscopy}
$\gamma$探测器的设置方式描述
\begin{enumerate}
    \item As shown in fig.2, a segmented germanium clover detector was placed at 0 degree, a few centimeters away from the silicon stack. To reduce the dead time of the acquisition system, the $\gamma$-ray signals were not used to trigger the data acquisition. 
\end{enumerate}

$\beta$-$\gamma$-$p$符合开窗确定探测效率
\begin{enumerate}
    \item As a consequence, the probability to observe a $\gamma$-decay depended on the type of radioactivity event that had triggered the acquisition system. Since the energy loss of a proton in \textit{E}4 was larger than a few hundred keV, a trigger signal was obtained each time a proton was emitted. Subsequent $\gamma$-rays were then automatically detected, depending only on the germanium detector efficiency.

    On the other hand, most of the $\beta$-particles emitted without accompanying protons did not lose enough energy in \textit{E}4 to trigger the acquisition system. As a consequence,the trigger efficiency for $\beta-\gamma $ events was given mainly by the fraction of the total solid angle under which the large silicon detector \textit{E}5 is seen from \textit{E}4. This efficiency was determined by means of $^{24}\textnormal{Al}$ which decays by $\beta-\gamma $ emission. The absolute intensities of the two main $\gamma$ lines at 1077 and 1369 keV were measured and compared with the expected values. The $\beta$-trigger rate was then derived, taking into acccount the intrinsic efficiency of the germanium detector, which was obtained with the conventional calibration sources. The overall $\gamma$ detection efficiency in the 300 to 2000 keV range was about 2 to 3\%, with a relative uncertainty of about 20\%. The $\beta$-trigger efficiency was equal to 35.0(45)\%.

    To a large extent, corrections due to true summing effects were included in the calculated $\beta$-trigger rate. However, this effect was not under control when the acquisition was triggered by the detection of protons, where the trigger efficiency was 100\%. Hence, $\gamma$-ray intensities could not be determined reliably for $\beta-p-\gamma $ decay events and therefore $\beta$-decay branching ratios towards proton-emitting states could not be cross-checked by means of $\gamma$ spectroscopy.
\end{enumerate}

\section{Experimental results}

$\beta$衰变属性与先前的研究进行了对比。
\begin{enumerate}
    \item The $\beta$-decay properites of $^{25}\textnormal{Si}$ are compared in the following to the results obtained in previous work.
\end{enumerate}
给出了绝对的分支比。
\begin{enumerate}
    \item For the two settings on $^{25}\textnormal{Si}$ and on $^{26}\textnormal{P}$, the relative intensities of the identified proton groups are given as well as the deduced absolute $\beta$-decay branching ratios towards the proton-unbound nuclear states of the daughter nuclei.
\end{enumerate}
测得 $\gamma$谱并进行处理。
\begin{enumerate}
    \item The analysis of $\beta$-delayed $\gamma$ spectra gives rise, for the first time, to the measurement of the absolute feeding of the proton-bound states.
\end{enumerate}
获得衰变纲图,并使用??相互作用计算。
\begin{enumerate}
    \item The decay schemes are then proposed and compared to calculations performed in the full \textit{sd} shell by Brown with the OXBASH code using the USD interaction.
\end{enumerate}
GT强度分布比较、半衰期测量、质量过剩。
\begin{enumerate}
    \item  Finally, The Gamow-Teller strength distributions are compared to those extracted from the calculated $\textnormal{log}\left( ft \right)$ values. The main characteristics of $^{26}\textnormal{P}$ are given, including a precise measurement of its lifetime as well as a derivation of its proton separation energy $S_{p}$ and of its atomic mass excess $\Delta \left( ^{26}\textnormal{P} \right)$.

\end{enumerate}

\subsection{$\beta$-decay study of $^{25}\textnormal{Si}$}
\subsubsection{$\beta$-decay proton emission}
汇总:给出质子谱,并总结哪些峰在之前文章中已经被发现。列出能量和强度的对比表格。
\begin{enumerate}
    \item The $\beta$-delayed proton emission spectrum obtained in coincidence with the detector \textit{E}5 is shown in fig.6 for the setting on $^{25}\textnormal{Si}$. Most of the proton groups reported in previous work by Robertson \textit{et al.} and Hatori \textit{et al.} have been identified. Their center-of-mass energies and their relative intensities are compared in table 2 and discussed in the following.
\end{enumerate}

解释未找到的质子原因
\begin{enumerate}
    \item \textit{Missing transitions:} Nine of the thirty-two proton groups reported by Robertson \textit{et al.} were not observed in the present work. Six of these transitions (see table 2) were not observed in the work of Hatori \textit{et al.}, too, and it is therefore plausible that they are due to the decay of $\beta-p$ contaminants in the experiment performed by Robertson \textit{et al.} The Three remaining missing transitions have a relative intensity lower than 1\% and it may be that the residual $\beta$ background in the spectrum conditioned by \textit{E}5 is too large in the present experiment to allow for their identification.
\end{enumerate}

解释观测的质子谱,并对部分偏离较大的质子峰解释。
\begin{enumerate}
    \item \textit{Identification of the observed proton groups:} All the observed proton groups were attributed to proton transitions reported in the work of Robertson \textit{et al.} on the basis of their measured center-of-mass energies. Two groups at 2980(9) and 3899(2) keV were tentatively identified as being the same transitions as those at 3021(9) and 3864(20) keV by Robertson \textit{et al.} although the energy differences are about 40 keV. The proton group at 401(1) in the present work corresponds most likely to the 382(20) keV group of Robertson \textit{et al.} because of their high relative intensities. The transition at 1377(6) keV was attributed to an emission from the IAS of $^{25}\textnormal{Al}$ and was therefore identified as the transition at 1396(20) keV of Robertson \textit{et al.} The same is most likely ture for the transitions at 1573(7) and 1592(20) keV. The proton group at 3326(6) keV was observed at the same energy as in the work of Hatori \textit{et al.} and corresponds most likely to the transition at 3342(15) keV in ref.26. All other transitions identified were measured at energies differing by less than 15 keV with respect to the work of Robertson \textit{et al.}
\end{enumerate}

解释高能区未观测到几个质子峰的原因,以及探测效率的问题。
\begin{enumerate}
    \item \textit{High-energy proton groups:} Only one of the three high-energy transitions reported by Zhou \textit{et al.} was identified at 6802(7) keV. Its relative intensity of 2.2(5)\% is significantly higher than the values given in refs. and, which might reflect an underestimation of the proton detection efficiency at high energies in the present work.
\end{enumerate}

观测到了新的质子峰,但是无法指认。原因是统计太低,而且无法通过$\beta -p-\gamma $来确认。
\begin{enumerate}
    \item \textit{New transition:} A new proton transition at 3077(14)keV (label 13) was observed but could not be attributed. Due to its low intensity of 0.25(11)\%, the transition could not be assigned neither by means of a $\beta-p-\gamma $ coincidence nor by any other means.
\end{enumerate}

质子峰的指认以及与先前工作的对照:
\begin{enumerate}
    \item \textit{Assignment of proton transitions:} Apart from the transition at 3326 keV which, according to Hatori \textit{et al.}, originates from the 5597 keV excited level in $^{25}\textnormal{Al}$, all identified proton groups were assigned following the work of Robertson \textit{et al.} The deduced energies and absolute $\beta$-decay branching ratios of the proton-unbound states of $^{25}\textnormal{Al}$ are presented in table 3. The obtained excitation energies are compared to the data of the compilation. Large discrepancies of more than 25 keV are found for the proton groups at 2980,3899 and 5407 keV. The IAS of $^{25}\textnormal{Al}$ was found at an excitation energy of 7892(2) keV, in agreement with the value of 7896(6) reported by Robertson \textit{et al.} 
    
    The overall agreement between the three $\beta$-delayed proton decay studies of $^{25}\textnormal{Si}$ is reasonable, leading to a summed $\beta$-decay branching ratio towards the proton-unbound states of $^{25}\textnormal{Al}$ equal to 35(2)\% (this work), 38(2)\% and 41(1)\%. The difference originates for a large part from the determination of the absolute intensity of the large energetic proton group at about 400 keV in the center of mass. This proton transition is reported in the previous work to compete with a $\gamma$ deexcitation of the associated nuclear state, but no evidence was found in the $\gamma$-decay spectrum for such a decay mode.

    Regrading the absolute $\beta$ feeding of the IAS in $^{25}\textnormal{Al}$, the value of 12.8(8)\% obtained in this work is in good agreement with the theoretically expected value of 12.2\% used by Robertson \textit{et al.} and is significantly lower than the one measured by Hatori \textit{et al.} It leads to a $\textnormal{log}\left( ft \right)$ value of 3.25(3) for the $\beta$-decay of $^{25}\textnormal{Si}$ towards the IAS in $^{25}\textnormal{Al}$. This result confirms the assumption that the involved $\beta$ transition is almost purely of the Fermi type, since a $\textnormal{log}\left( ft \right)$ value of 3.28 is expected in this case.
\end{enumerate}



\subsubsection{$\beta$-delayed $\gamma$-decay}
给出 $\gamma$ 谱,并进行描述。针对特殊情况进行解释。
\begin{enumerate}
    \item The $\gamma$-ray spectrum obtained in the decay of $^{25}\textnormal{Si}$ is shown in fig.7. The four $\gamma$ lines at 452 (absolute branching ratio of 18.4(42)\%), 493(15.3(34)\%), 945(10.4(23)\%) and 1612 keV (15.2(32)\%) were assigned to the $\beta$-delayed $\gamma$-decay of $^{25}\textnormal{Si}$. The last $\gamma$ line is a doublet of two $\gamma$-rays from the decay of the $\frac{7}{2}_{1}^{+}$ states at 1612.4 keV in $^{25}\textnormal{Al}$ and at 1611.7 keV in its daughter nucleus $^{25}\textnormal{Mg}$. Taking into acccount the expected contribution of this second transition in $^{25}\textnormal{Mg}$, the absolute intensity of the 1612 keV $\gamma$-ray in $^{25}\textnormal{Al}$ was deduced to be equal to 14.7(32)\%.
\end{enumerate}

能够通过质子关联的 $\gamma$ 分支指认。
\begin{enumerate}
    \item The $\gamma$ lines at 493 and 945 keV are associated with the de-excitation of the $\frac{3}{2}_{1}^{+}$ state at 945 keV in $^{25}\textnormal{Al}$ towards its $\frac{5}{2}_{1}^{+}$ ground state and towards the $\frac{1}{2}_{1}^{+}$ excited state at 452 keV. The intensity ratio of the two lines $I_{\gamma}(945)/I_{\gamma }(493)=68(26)\%$ is in agreement with the value of 79(6)\% obtained in an in-beam experiment.
\end{enumerate}

排除部分未关联 $\gamma$
\begin{enumerate}
    \item Since the intensities of the 493 and 452 keV $\gamma$-rays were found to be equal within their uncertainties, we conclude that 452 keV state is not fed directly in the $\beta$-decay of $^{25}\textnormal{Si}$. Such a $\beta$ transition would be indeed a first-forbidden one is therefore unlikely to be observed in the present experiment.
\end{enumerate}

未布局(未被观察到)的 $\gamma$ 会造成附近的质子造成$\beta$叠加效应。
\begin{enumerate}
    \item No $\gamma$-rays were observed at 845, 1338 and 1790 keV. Therefore, it was assumed that the $\frac{7}{2}_{2}^{+}$ proton-bound state of $^{25}\textnormal{Al}$ at 1790 keV is not fed in the $\beta$-decay of $^{25}\textnormal{Si}$. Hence, the measurements of the absolute intensities of the three $\gamma$ lines at 493, 945 and 1612 keV led to a summed $\beta$-decay branching ratio towards the proton-bound excited states of $^{25}\textnormal{Al}$ of 41(5)\% (see table 4 for details). Taking into account the previously determined summed $\beta$-decay branching ratio towards the proton-unbound states(35(2)\%), this leads to an absolute $\beta$ feeding of the $^{25}\textnormal{Al}$ ground state of 25(7)\%.
\end{enumerate}

$\beta$-\textit{p}-$\gamma$ $\gamma$指认
\begin{enumerate}
    \item The $\gamma$ line observed at 1369 keV corresponds to the deexcitation of the first-excited state of $^{24}\textnormal{Mg}$ populated in the $\beta-p$ decay of $^{25}\textnormal{Si}$. Due to the quite low $\gamma$ detection efficiency and the weakness of most of the proton transitions feeding excited states of $^{24}\textnormal{Mg}$, neither the $4_{1}^{+}\to 2_{1}^{+}$ nor the $2_{2}^{+}\to 4_{1}^{+}$ transitios were seen. Only a few counts at an \textit{E}4 energy of about 4.25 MeV were observed in coincidence with the $\gamma$-ray at 1369 keV, in agreement with the assignment of the strongest proton group to the IAS in $^{25}\textnormal{Al}$. The $\gamma$ line at 1461 keV is the well-known background $\gamma$-ray from $^{40}\textnormal{K}$.
\end{enumerate}

\subsubsection{$\beta$-decay scheme of $^{25}\textnormal{Si}$}
实验和理论计算的能级纲图对比,符合情况
\begin{enumerate}
    \item Figure 8 shows the $\beta$-decay scheme proposed for $^{25}\textnormal{Si}$. The experimental branching ratios and the corresponding $\textnormal{log}\left( ft \right)$ values are compared to shell-model calculations performed by Brown. Only excited states predicted to be fed with a branching ratio of more than 0.1\% are taken into account. In terms of nuclear structure, the agreement between experiment results and theoretical calculations appears to be very good, most of the observed nuclear states being reproduced by the model within a few hundred keV.
\end{enumerate}

GT强度对比
\begin{enumerate}
    \item The summed Gamow-Teller strength distribution as a function of the excitation energy of $^{25}\textnormal{Al}$ is shown in fig.9. The experimental distribution is in good agreement with the one deduced from the shell-model calculations up to 6 MeV. Beyound, the model predicts the feeding of a lot of high-energy excited states by low-intensity $\beta$ transitions that are not visible experimentally. Due to the small phase space factor \textit{f} associated with such transitions,the related \textit{B}(GT) values are of importance, which explains the divergence at more than 6 MeV of excitation energy. The global agreement below 6 MeV is obtained for 11 indications $\beta$ transitions for which the Gamow-Teller strength is quenched equivalent to a quenching factor of about 0.6.
\end{enumerate}


镜像对称性的描述
\begin{enumerate}
    \item At low excitation energy, the Gamow-Teller strength seems to be close to the one expected from the $\beta$-decay of the $^{25}\textnormal{Si}$ mirror nucleus, assuming that nuclear forces are isospin independent ($\delta=0$). Unfortunately, the error for the $\beta$-decay branching ratios towards these states is too large (see table 8 below) due to the uncertainty on the $\gamma$ detection efficiency and the individual values of the asymmetry parameters $\delta$ could not be derived precisely for the (\textit{A}=25,\textit{T}=3/2) isospin multiplet ($^{25}\textnormal{Na}$, $^{25}\textnormal{Mg}$, $^{25}\textnormal{Al}$, $^{25}\textnormal{Si}$).
\end{enumerate}

\subsection{$\beta$-decay study of $^{26}\textnormal{P}$}

\subsubsection{$\beta$-delayed proton emission}
\subsubsection{$\beta$-delayed $\gamma$-decay}
\subsubsection{Measurement of the half-life of $^{26}\textnormal{P}$}
半衰期的确定方式
\begin{enumerate}
    \item The lifetime of $^{26}\textnormal{P}$ was determined by means of a time corrlation procedure. The applied techiques is schematically shown in the inset of fig.12. It consists in measuring the time difference between the implantation of $^{26}\textnormal{P}$ ions, identified by means of time-of-flight and energy loss measurements, and the observation of $\beta$ or $\beta-(2)p$ decay events.
\end{enumerate}

长寿命本底以及其他可能的污染物的描述、确定的半衰期及与先前工作的对比
\begin{enumerate}
    \item Decay events that are correlated to the selected implantation event follow an exponential decay curve, whereas uncorrelated events due to the decay of contaminants ions, due to $^{26}\textnormal{P}$ daughter nuclei or due to $^{26}\textnormal{P}$ implantations other than the one considered for the correlation are randomly distributed. The large time corrlation window of 500 ms enabled us to estimate accurately the contribution of uncorrelated events to the decay curve. The half-life of $^{26}\textnormal{P}$ was measured to be 43.7(6) ms, in agreement with the value given by Cable \textit{et al.} of $20_{-15}^{+35}$ ms. We verified that, due to its relatively long half-life (2.21 s), the daughter decay of $^{26}\textnormal{Si}$ does not alter the fit result.
\end{enumerate}

\subsubsection{$\beta$-decay scheme of $^{26}\textnormal{P}$}
\subsection{Mirror asymmetry of mass A=25,26 nuclei}
镜像核非对称性随能级变化的规律。
\begin{enumerate}
    \item The mirror asymmetry parameter $\delta$ is usually determined for the ground-state transitions as well as for those feeding the low-lying excited states in the daughter nuclei. Higher-lying states are normally fed with smaller branching ratios, which yields larger errors for the $\delta$ value. In additon, these states may decay by proton emission for the proton-rich partner which usually reduces the branching ratio precision.
\end{enumerate}

In the present experiment, however, the feeding of low-lying states and in turn also of the ground state (its branching ratio is determined as the difference between 100\% and the observed branchings) is only poorly determined due to the large uncertainties of the $\gamma$-ray efficiency of our set-up. Nontheless, we give the asymmetry values derived from the present work for the mass \textit{A}=25 and \textit{A}=26 nuclei in table 8.

Experimentally, we reach the best precision for the highest-lying state in each mirror couple where the \textit{ft} value for the proton-rich nucleus comes from a $\beta$-delayed proton branch. In both cases, a significant effect is observed. This result,however, is opposite in sign compared to the theoretical value for the mass \textit{A}=25 couple. For the other mirror transitions, no clear statement can be made due to the large experimental errors for the decay of the proton-rich partner.
\section{Conclusion and perspectives}
The $\beta$-decay of the neutron-deficient nuclei $^{25}\textnormal{Si}$ and $^{26}\textnormal{P}$ was studied at the LISE3 facility at GANIL. 300 and 60 ions per second, respectively, were produced with contamination rates of less than 1\% and of about 13\%. The decay scheme of two nuclei was obtained, including for the first
time the $\beta$-decay pattern towards proton-bound states. It allowed us to measure the asymmetry parameter $\delta$ for the mirror states of the mass \textit{A}=25 and \textit{A}=26 nuclei. Unfortunately, the poor precision in the determination of the corresponding branching ratios gave rise to large uncertainties for these $\delta$ values. The comparison to shell-model calculations based on the USD interaciton and performed in the full \textit{sd} shell by Brown revealed two features: the reliability of such models when they are applied to mid-shell nuclei lying close to the proton drip line, and the about 60\% quenching of the Gamow-Teller strength of the individual $\beta$ transitions.

The following properties were derived from the spectroscopic study of this nuclei.
\begin{itemize}
    \item[i)] The half-life of $^{26}\textnormal{P}$ was measured to be equal to 43.7(6) ms.
    \item[ii)] Its proton separation energy as well as the maximum available energy in its $\beta$-decay were determined with a precision of 90 keV. 
    \item[iii)] The $\beta$-delayed two-proton emission of $^{26}\textnormal{P}$ towards the ground state and the first-excited state of $^{24}\textnormal{Mg}$ was observed.
    \item[iv)] More than thirty one-proton groups were identified,five of them being emitted from the isobaric analog state of $^{26}\textnormal{Si}$.    
\end{itemize}

Compared to previous studies with a helium-jet technique, the use of projectile fragmentation in conjunction with a fragment separator has several advantages, which are that:
\begin{itemize}
    \item[i)] the detection of the arrival of an ion allows its identification and gives a start signal for half-life measurements
    \item[ii)] very short half-lives can be studied since the separation time is short (of order 1 microsecond)
    \item[iii)] the selection process is independent of chemistry. 
\end{itemize}

Nontheless, the spectroscopic studies presented here suffer from limitations that should be addressed in future experiments of the same type. Firstly, the implantation of ions inside a silicon detector gives rise to a high proton detection efficiency; however, due to the energy deposit of $\beta$-particles in the implantation detector, it is sometimes difficult to observe $\beta$-delayed protons with low intensity.

Secondly, concerning the $\gamma$ spectroscopy part, a high efficiency is required in order to identify low-intensity $\gamma$-rays,and  a high precision is needed for the more intense transitions. A new detection set-up using segmented silicon detectors and four germanium clovers has therefore been implemented and the decay properites of $^{21}\textnormal{Mg}$, $^{25}\textnormal{Si}$ and their mirror nuclei were investigated recently at the GANIL facility. This work should lead to the determination of accurate asymmetry parameters $\delta$, which might help to understand the origin of isospin non-conserving forces in nuclei.

The author would like to thank B.A. Brown for providing up-to-date shell-model calculations and C. Volpe and N.A. Smirnova for stimulating discussions about the mirror asymmetry question. We would like to acknowledge the continuous effort of the whole GANIL staff for ensuring a smooth running of the experiment. This work was supported in part by the Conseil Regional d'Aquitaine.
\end{document}