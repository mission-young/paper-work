\documentclass[UTF8]{ctexart}
\usepackage{xcolor}
% \usepackage{amsmath}
\begin{document}
\title{题目}
\thanks{脚注}
\author{作者}
\author{二作}   
\author{三作者}
\date{\today}

\begin{abstract}
    \textbf{Abstract:}
    \newline
    {\color{red}?} 核的 {\color{red}$\beta$衰变} 属性在 {\color{red}实验地点}通过 {\color{red}带电粒子和 $\gamma$谱学} 方式被调研.
    \begin{enumerate}
        \item The $\beta$-decay properties of the neutron-deficient nuclei {\color{red}$^{25}$Si and $^{26}$P} {\color{blue}$^{22}$Al} have been investigated at {\color{red}the GANIL/LISE3 facility} {\color{blue}Radioactive Ion Beam Line in Lanzhou (RIBLL)} by means of charged-particle and $\gamma$-ray spectroscopy.
    \end{enumerate}
    {\color{red}能级纲图和跃迁强度核心的实验和理论对比描述。}
    \begin{enumerate}
        \item  The decay schemes obtained and the Garrow-Teller strength distributions are compared to shell-model calculations based on the USD interaction. 
    \end{enumerate}
    {\color{red}实验结果给出的一些信息}
    \begin{enumerate}
        \item $B(GT)$ values derived from the absolute measurement of the $\beta$-decay braching ratios give rise to a quenching factor of the Gamow-Teller strength of {\color{red}0.6}{\color{blue}null}. 
    \end{enumerate}
    {\color{red} 半衰期和衰变模式}
    \begin{enumerate}
        \item A precise half-life of {\color{red}43.7(6)}{\color{blue}null} ms was detemined for {\color{red}$^{26}$P}{\color{blue}$^{22}$Al},the $\beta$-(2)$p$ decay mode of which is described.
    \end{enumerate}
    \begin{description}
        \item[Usage] usage
        \item[PACS number] number
        \item[structure]   struct
    \end{description}

\end{abstract}
\maketitle

\section{introduction}
\subsection{generalities}
$\star\star\star\star ~background~description~ \star\star\star\star$

过去很多年, {\color{red}?}核的 {\color{red}?}属性已经被调研以验证??
\begin{enumerate}
    \item Over the last decades, $\beta$-decay properties of light unstable nuclei have been extensively investigated in order to probe their single-particle nuclear structure and to establish the proton and neutron drip lines.
\end{enumerate}

$\star\star\star\star ~shell ~model~  \star\star\star\star$

在非稳定核区域,$\beta$衰变可以用来检验这些模型。

Hence, compilation of spectroscopic properites are available for many $sd$ shell nuclei from which nucleon-nucleon interactions were derived. $\beta$-decay studies of nuclei having a large proton excess are therefore useful to test the validity of these models when they are applied to very unstable nuclei.

$\star\star\star\star ~reduced ~transition ~probability ~ft \star\star\star\star$

约化跃迁强度ft用于描述实验结果和弱相互作用基本参数的物理量。

Moreover, in the standard V-A description of $\beta$-decay, a direct link between experiment results and fundamental constants of the weak interaction is given by the reduced transition probability $ft$ of the individual allowed $\beta$-decays. This parameter,which incorporates the phase space factor $f$ and the parital half-life $t=T_{1/2}/\rm{B.R}$. ($T_{1/2}$ being the total half-life of the decaying nucleus and B.R. the branching ratio associated with the $\beta$ transition considered), can be written as follow:

$\star\star\star\star~equation~ft~\star\star\star\star$

\begin{equation}
    ft=\frac{\kappa}{g^{2}_{V}\Big|\langle f|\tau|i \rangle\Big|^{2}+g^{2}_{A}\Big|\langle f|\sigma\tau|i \rangle\Big|^{2}}
\end{equation}

$\star\star\star\star~ft~explanation~\star\star\star\star$

where $\kappa$ is a constant and where $g_{V}$ and $g_{A}$ are, respectively, the vector and axial-vector current coupling constants related to the Fermi and Gamow-Teller components of $\beta$-decay. $\tau$ and $\sigma$ are the isospin and the spin operators, respectively. 

$\star\star\star\star~ft~meanings~\star\star\star\star$

测量所得的ft值与Fermi和GT计算所得的比较,用于检验壳模型波函数,可以看出原子核初末态波函数重叠以及父态、子态的组态混合。

Hence, the comparison of the measured $ft$ values and the computed Fermi and Gamow-Teller matrix elements appears to be a good test of nuclear wave fucntions built in the shell-model frame, stressing the role of the overlap between initial and final nuclear states as well as the configuration mixing occurring in parent and daughter states. 

$\star\star\star\star~mirror~asymmetry~anomaly~and~quenching~of~the~GT~strength~\star\star\star\star$

两处系统性地偏离理论预测的情形体现了理论理解和基础相互作用认知的局限,即 $\beta$衰变同位旋对称性破缺以及GT强度的猝灭。

However, two systematic deviations from theoretical predictions shows the limitation of our theoretical understanding and treatment of fundamental interactions. They are reported as the \textit{mirror asymmetry anomaly in $\beta$} and the \textit{quenching of the Gamow-Teller strength}.

$\star\star\star\star~Mirror~asymmetry~in~\beta~decay~\star\star\star\star$

$\beta$衰变同位旋破缺的描述和理解。

\textit{Mirror asymmetry in $\beta$-decay}: This phenomenon is related to the isospin non-conserving forces acting in the atomic nucleus. If nuclear forces were charge independent, the $\beta^{+}$(EC) and $\beta^{-}$ decays of analog states belonging to mirror nuclei would be of equal strength. The deviation from this simple picture is characterized by the asymmetry parameter $\delta=\left( ft^{+}/ft^{-}-1 \right)$, where the + and - signs are associated with the decay of the proton- and the neutron- rich members of the mirror pair, respectively. Figure 1 presents an updated systematic of $\delta$ values measured for mirror nuclei with A $\leq$ 40. Thirty-nine allowed Gamow-Teller mirror transitions with log(\textit{ft}) $\leq$ 6 pertaining to 14 pairs of a mean deviation of about 5\% for these nuclei lying in the $p$ and $sd$ shells. The asymmetry reaches 11(1)\% if only $p$ shell nuclei are considered, which stresses the interplay between the Coulomb and the centrifugal barriers.

对称性破缺的两种理论解释。次级电流和波函数重叠。

It was often attempted to explain the mirror asymmetry anormaly in the $p$ shell either in terms of binding effects or by troducing the concept of " second-class currents ", which are not allowed within the frame of the standard V-A model of the weak interaction. None of the theoretical approaches ware able to reproduced the measured $\delta$ values. Shell-model calculations are currently performed to test the isospin non-conserving part of the interaciton in $\beta$-decay by studying the influence of isospin mixing effects and of radial overlap mismatches of nuclear wave functions on the Gamow-Teller matrix elements. These calculations are performed in the $p$ shell and in the $sd$ shell, where reliable single-particle nuclear wave functions are now available.


$\star\star\star\star~Gamow-Teller~quchching~\star\star\star\star$



\textit{Gamow-Teller quenching:} The axial-vector couping constant $g_{A}$ involved in $\beta$ transitions of the Gamow-Teller type is not strictly constant and it has to be renormalized in order to reproduce the $ft$ values measured experimentally. The effective couping constant $g_{A,eff}=q\cdot g_{A}$ is deduced empirically from nuclear-structure experiments and shows a slight variation over a wide range of masses: $q=0.820(15)$ in the $p$ shell,$q=0.77(2)$ in the $sd$ shell(giving a quenching factor $q^{2}$ of 0.6) and $q=0.744(15)$ in the $pf$ shell.

Different theoretical approaches haves been used in order to derive the renormalization factor from core polarization effects (due to particle-hole excitations), isobar currents and meson exchange. Despite all these efforts, the origin of the quenching effect is not very well understood. Nevertheless, the Gamow-Teller strength function $B(GT)=\left( g_{A}/g_{V} \right)^{2}\left| \sigma \tau  \right|^{2}$, which translates the global response of the wave function to spin-isospin excitations ocuurring in the $\beta$-decay,is a useful link between experimental results and theoretical predictions and it can be used as a comparative tool.

$\star\star\star\star~Experimental~development~\star\star\star\star$



\textit{Experimental development:} With the development of secondary radioactive beams and other experimental techiques like the combination of helium-jet transport systems with telescope detectors, a large set of neutron-deficient nuclei has been investigated since the $\beta$-decayed proton emission was first first observed forty years ago. As $Q_{EC}$ values are increasing while nuclei become more exotic, $\beta-p$ and $\beta-\gamma$ spectroscopic studies of neutron-deficient nuclei give the opportunity to probe the Gamow-Teller strength function up to more than 10 MeV in excitation energy. Hence, the whole energy window open in $\beta$-decay can be covered both by spectroscopic studies and charge exchange reactions. Therefore, the theoretical description of nuclear structure as well as our understanding of the weak interaction can be tested far from the stability line. As an illustration,we will report in the following on the $\beta$-decay properties of two neutron-deficient light nuclei, namely $^{25}\textnormal{Si}$ and $^{26}\textnormal{P}$.

\subsection{Previous studies}
\subsubsection{Studies of $^{25}\textnormal{Si}$}
描述先前测得核的半衰期,$Q_{\textnormal{EC}}$,同位旋等信息。
\begin{enumerate}
    \item With a lifetime of 218 ms and a $Q_{\textnormal{EC}}$ value of about 13 MeV, the $T_{Z}=-\frac{3}{2}$ nucleus $^{25}\textnormal{Si}$ has been studies several times since the end of the 1960s. 
\end{enumerate}

描述先前实验在本次实验中承担的作用,分支比对比,刻度等等。
\begin{enumerate}
    \item These previous studies will be used in the present work to validate the analysis procedure implemented to derive the $\beta$-decay properties of $^{26}\textnormal{P}$. 
\end{enumerate}

描述先前实验中不足的地方。如缺乏某类灵敏探测器。探测器分辨不足,立体角覆盖小等等。
\begin{enumerate}
    \item However, none of these studies measured the decay by $\gamma$ emission of excited stated fed in the $\beta$-decay of $^{25}\textnormal{Si}$.
\end{enumerate}

描述下最近的一个实验做的事情。

\begin{enumerate}
    \item The most recent $\beta$-delayed proton emission study of $^{25}\textnormal{Si}$ was performed by Robertson \textit{et al.} It updates the first investigation of Reeder \textit{et al.} in 1966. In both experiments, the individual proton group intensities were measured relative to the most intense one, emitted by the isobaric analog state (IAS) in $^{25}\textnormal{Al}$. The absolute $\beta$-decay branching ratio of 12.2\% towards this state was derived from the associated $\textnormal{log}\left( ft \right)$ value $\left( \textnormal{log}ft=3.28 \right)$, calculated assuming a pure Fermi $\beta$ transition from the ground state of $^{25}\textnormal{Si}$. It led to a summed $\beta$ feeding of proton-unbound states of $^{25}\textnormal{Al}$ equal to 28.1(15)\%. This normalization procedure is supported by the measurement of Hatori \textit{et al.}. 
\end{enumerate}

本工作和上述工作的对比,是否一致。

\begin{enumerate}
    \item In this work, absolute branching ratios for $\beta$-decay were determined by counting the total number of $\beta$-particles emitted with the half-life of $^{25}\textnormal{Si}$ and the $\beta$ feeding of the IAS in $^{25}\textnormal{Al}$ was indeed found to be equal to 14.6(6)\%, giving raise to a $\textnormal{log}ft$ value of 3.19(2). The summed feeding of the $^{25}\textnormal{Al}$ proton-emitting states was measured to be 40.7(14)\%, in good agreement with Robertson \textit{et al.}
\end{enumerate}

总结一下工作中的不足,以及可能带来的后果。
\begin{enumerate}
    \item As mentioned above, in none of the experiments, the $\beta$-delayed $\gamma$-decay of $^{25}\textnormal{Si}$ was observed. As a consequence, the $\beta$-decay branching ratios towards the proton-bound states of $^{25}\textnormal{Al}$ were tentatively estimated taking into account the summed $\beta$ feeding and assuming that the relative \textit{ft} values of these states were equal to those of the mirror states in $^{25}\textnormal{Mg}$. The weak point of such a procedure is that an average $\beta$ asymmetry of 20\% had to be taken into account for all proton-bound states, which was assumed to be equally shared by the proton-bound states disregarding their individual quantum characteristics.
\end{enumerate}

\subsubsection{Studies of $^{26}\textnormal{P}$}

Due to its $T_{Z}$ value of -2 and its short lifetime of less than 100ms, $^{26}\textnormal{P}$ has not been investigated in detail so far. Compilations only report the observation by Cable \textit{et al.} of $\beta$-delayed proton and two-proton emission from this nucleus. A half-life of $20_{-15}^{+35} \textnormal{ms}$ was deduced from the observation of the most intense proton group. It led to a $\beta$ feeding of the IAS in $^{26}\textnormal{Si}$ equal to $1.9_{-3.5}^{-1.4}\%$ using a calculated $\textnormal{log}\left( ft \right)$ value of 3.19 (assuming a pure Fermi transition). Only three proton groups were observed linking the IAS to the two lowest states of $^{25}\textnormal{Al}$ ($\beta-p$ decay) and to the ground state of $^{24}\textnormal{Mg}$ ($\beta-2p$ decay). The two decay modes of the IAS were reported to be of similar magnitude. However, the large $Q_{\textnormal{EC}}$ value of 18 MeV together with a proton speration energy of 5.5 MeV for the daughter nucleus $^{26}\textnormal{Si}$ are indications that the $\beta$-delayed charged-particle spectrum may be rather complex, involving a large number of proton groups.

\subsubsection{Present measurement}

在这个实验中,通过??方式确定了绝对分支比。
\begin{enumerate}
    \item In our experiment, we determined the absolute branching ratios for $^{25}\textnormal{Si}$ and $^{26}\textnormal{P}$ by relating the intensity of a given proton or $\gamma$ line to the number of isotopes of each type implanted in our set-up. 
\end{enumerate}
For $^{25}\textnormal{Si}$, this measurement consititude a first unambigous determination of braching ratios also for proton-bound levels. We will use the decay of $^{25}\textnormal{Si}$ in part to test our analysis procedure, however, our study yields also new results for this nucleus, in particular for the $\gamma$-decay of its $\beta$-decay daughter. In the case of $^{26}\textnormal{P}$, we deduce for the first time the feeding for other states than the IAS and their decay by proton or $\gamma$ emission. Therefore, we could establish a complete decay scheme for braches with more than about 1\% feeding for both nuclei for the first time.
\section{Experiment procedure}
\subsection{Fragment production and detection set-up}

描述除了目标核外还有哪些核也被产生,以及他们各自的目的。
\begin{enumerate}
    \item In additon to $^{25}\textnormal{Si}$ and $^{26}\textnormal{P}$, the $\beta$-delayed proton and two-proton emitters $^{22}\textnormal{Al}$ and $^{27}\textnormal{S}$ have been studied during the same experimental campaign. The $\beta$-delayed proton emitter $^{21}\textnormal{Mg}$ and the $\beta$-delayed $\gamma$ emitter $^{24}\textnormal{Al}$ were also produced for calibration and efficiency measurement purposes.
\end{enumerate}

描述束流强度,种类以及设施。
\begin{enumerate}
    \item All nuclei have been produced in the fragmentation of a 95 MeV / \textit{u} $^{26}\textnormal{Ar}^{18+}$ primary beam with an intensity of about 2 $\mu$A$e$ delivered by the coupled cyclotrons of the GANIL facility.
\end{enumerate}

描述靶材设置以及其他束流选取的设置。
\begin{enumerate}
    \item A 357.1mg/cm$^{2}$ $^{12}\textnormal{C}$ production target was placed in the SISSI device, the high angular acceptance and focusing properites of which increased the selectivity of the fragment separation operated by the LISE3 spectrometer. The latter included a shaped Be degrader (thickness 1062 $\mu$m) at the intermediate focal plane and a Wien filter at the end of the line to refine the selection of the separated fragments.
\end{enumerate}

感兴趣核注入位置描述。
\begin{enumerate}
    \item Ions of interest were implanted in the fourth element \textit{E}4 of a silicon stack.
\end{enumerate}

粒子鉴别方式。
\begin{enumerate}
    \item The ion identification was performed by means of time-of-flight and energy loss measurements with the silicon detecotrs \textit{E}1 to \textit{E}4 (2*300 $\mu$m and 2*500 $\mu$m in thickness, 4*600 mm$^{2}$ of surface).
\end{enumerate}

 It led to a precision in the counting rate of better than 1\% for $^{25}\textnormal{Si}$ and about 3\% for the more exotic $^{26}\textnormal{P}$ nucleus. The production method in association with the high selectivity of the LISE3 spectrometer gave rise to a very low contamination rate of the selected species by only a few isotones.

探测器之间的符合关联
\begin{enumerate}
    \item Protons were detected in the implantation detector \textit{E}4, in coincidence with the observation of $\beta$-particles in the detector \textit{E}5 (with a thickness of 6 mm and an area of 600 mm$^{2}$). A segmented germanium clover was finally used to study the $\beta$-delayed $\gamma$-decay of implanted ions.
\end{enumerate}

\subsection{$\beta$-delayed proton spectroscopy}
$\beta$叠加效应部分的描述
\begin{enumerate}
    \item Contrary to previous experiments in which ions were deposited at the surface of an ion cather, $\beta$-delayed protons are emitted inside the implantation detecotr \textit{E}4. As a first consequence, the proton spectrum rises on a large $\beta$ background and the identification of low-energy, low-intensity proton lines is difficult. Secondly, the energy deposit in the detector \textit{E}4 of an emitted proton cannot be disentangled from the energy loss contribution of the associated $\beta$-particle and the recoiling ion.
\end{enumerate}

减少 $\beta$叠加效应
\begin{enumerate}
    \item To minimize these effects, ions were implanted in the last 100 $\mu$m of the detector \textit{E}4 and a $\beta$ coincidence with the thicker detector \textit{E}5 was requested in the analysis. As shown in the upper part of fig.3, the $\beta$-particle energy deposit in the coincidence spectrum was strongly reduced and proton peaks could be easily identified and fitted with the help of Gaussian distributions. The energy calibration of the detector \textit{E}4 as well as the measurement of the proton group intensities were performed on the basis of this \textit{E}4-\textit{E}5 coincidence condition.
\end{enumerate}

\subsubsection{Energy calibration of the implantation detector}
$\beta$叠加效应正比于注入深度
\begin{enumerate}
    \item The $\beta$-particle energy deposit leads to a shift in energy of the Gaussian-like part of the proton peaks. This effect could be reproduced by means of a GEANT simulation, as shown in the lower part of fig.3 for a representative $\beta$-delayed proton peak. It could also be shown that the energy shift is roughly independent of the proton and $\beta$-particle energies but linearly dependent on the implantation depth of the ions, that is to say, on the distance the $\beta$-particles travel in the detector \textit{E}4 before leaving it to enter the coincidence detector \textit{E}5. The energy calibrations of the detector \textit{E}4 for the settings on $^{21}\textnormal{Mg}$,$^{25}\textnormal{Si}$ and $^{26}\textnormal{P}$ were therefore assumed to differ only by a shift proportional to the implantation depths of the ions.
\end{enumerate}

使用$^{25}\textnormal{Si}$的一系列质子峰来进行刻度。
\begin{enumerate}
    \item The calibration parameters for the settings were deduced from the identification of the major proton groups expected at 1315(9),1893(2),2037(4),2589(9),4908(3) and 6542(3) keV for the decay of $^{21}\textnormal{Mg}$ and at 402(1),1925(3),2169(7),2312(4),3472(10),4261(2) and 5630(2) keV for the decay of $^{25}\textnormal{Si}$. The proton group energies were recalculated using the excitation energies of the proton-emitting states and the proton separation energies reported in a compilation.
\end{enumerate}

\subsubsection{Proton detection efficiency}
描述影响探测效率的因素。以及Geant4模拟的效果。

\begin{enumerate}
    \item Since ions were implanted at the end of the detector \textit{E}4, the proton detection efficiency $\varepsilon_{p}$ is very sensitive to the implantation profile of the emitting ion and to the proton energy. The detection efficiency for protons between 0.5 and 10 MeV was computed by means of GEATN4 simulations. Following experimental observations, implantation profiles were approximated by Gaussian distributions in the beam direction (with a standard deviation of 20 $\mu$m) and with a two-dimensional square-shaped function in the orthogonal plane. 
    
    Results are shown in fig.4. An uncertainty on the detection efficiency of less than 6\% was obtained. This uncertainty was determined by varying the implantation depth by $\pm$10 $\mu$m, which is roughly the width of the implantation distribution.
\end{enumerate}

\subsubsection{Absolute intensities of the observed proton groups}
The absolute intensity $I_{p}^{i}$ of a given proton group \textit{i} was derived from the following relation:
\begin{equation}
    I_{p}^{i}=\frac{S_{c_{p}^{i}}}{K_{c_{p}} \cdot N_{\textnormal{impl}}\cdot \varepsilon_{p}^{i}}
\end{equation}
where $S_{c_{p}^{i}}$ is the area of the proton peak observed in the coincidence spectrum (\textit{E}5>0),$K_{c_{p}}$ the normalization factor to be taken into account due to the coincidence condition, $N_{\textnormal{impl}}$ the number of ions implanted in \textit{E}4 and $\varepsilon_{p}^{i}$ the proton detection efficiency for a given proton energy.

The extraction of the factor $K_{c_{p}}$ is illustrated in fig.5 for the setting on $^{25}\textnormal{Si}$. Several proton peaks were fitted in the high-energy part of the \textit{E}4 energy spectrum, where the $\beta$ background is low enough and where proton peaks are well separated. The $K_{c_{p}}$ coefficients were deduced from the average ratio of the areas of the $\beta$-delayed proton peaks obtained with and without coincidence condition. For the coincidence spectrum, peaks were fitted by means of Gaussian distributions on a linear background (see inserts of fig.5) leading to the $S_{c_{p}^{i}}$ values. For the unconditioned energy spectrum, fit fucntios convoluting a Gaussian distribution and an exponential tail on top of an exponential background were used. For each ion of interest, the parameters of the exponential tail were fixed regardless of the proton peak energies. The $K_{c_{p}}$ coefficients obtained were about 13\%, with an uncertainty of 1 to 2\%.



\subsection{$\gamma$-ray spectroscopy}
As shown in fig.2, a segmented germanium clover detector was placed at 0 degree, a few centimeters away from the silicon stack. To reduce the dead time of the acquisition system, the $\gamma$-ray signals were not used to trigger the data acquisition. As a consequence, the probability to observe a $\gamma$
\section{Experimental results}
\subsection{$\beta$-decay study of $^{25}\textnormal{Si}$}
\subsubsection{$\beta$-decay proton emission}
\subsubsection{$\beta$-delayed $\gamma$-decay}
\subsubsection{$\beta$-decay scheme of $^{25}\textnormal{Si}$}
\subsection{$\beta$-decay study of $^{26}\textnormal{P}$}
\subsubsection{$\beta$-delayed proton emission}
\subsubsection{$\beta$-delayed $\gamma$-decay}
\subsubsection{Measurement of the half-life of $^{26}P$}
\subsubsection{$\beta$-decay scheme of $^{26}\textnormal{P}$}
\subsection{Mirror asymmetry of mass A=25,26 nuclei}
\section{Conclusion and perspectives}

\end{document}